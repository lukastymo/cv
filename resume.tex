\documentclass[]{rss}
\usepackage{lmodern}
\usepackage{amssymb,amsmath}
\usepackage{ifxetex,ifluatex}
\usepackage{fixltx2e} % provides \textsubscript
\ifnum 0\ifxetex 1\fi\ifluatex 1\fi=0 % if pdftex
  \usepackage[T1]{fontenc}
  \usepackage[utf8]{inputenc}
\else % if luatex or xelatex
  \ifxetex
    \usepackage{mathspec}
  \else
    \usepackage{fontspec}
  \fi
  \defaultfontfeatures{Ligatures=TeX,Scale=MatchLowercase}
\fi
% use upquote if available, for straight quotes in verbatim environments
\IfFileExists{upquote.sty}{\usepackage{upquote}}{}
% use microtype if available
\IfFileExists{microtype.sty}{%
\usepackage[]{microtype}
\UseMicrotypeSet[protrusion]{basicmath} % disable protrusion for tt fonts
}{}
\PassOptionsToPackage{hyphens}{url} % url is loaded by hyperref
\usepackage[unicode=true]{hyperref}
\PassOptionsToPackage{usenames,dvipsnames}{color} % color is loaded by hyperref
\hypersetup{
            pdftitle={Resume},
            colorlinks=true,
            linkcolor=blue,
            citecolor=Blue,
            urlcolor=Blue,
            breaklinks=true}
\urlstyle{same}  % don't use monospace font for urls
\IfFileExists{parskip.sty}{%
\usepackage{parskip}
}{% else
\setlength{\parindent}{0pt}
\setlength{\parskip}{6pt plus 2pt minus 1pt}
}
\setlength{\emergencystretch}{3em}  % prevent overfull lines
\providecommand{\tightlist}{%
  \setlength{\itemsep}{0pt}\setlength{\parskip}{0pt}}
\setcounter{secnumdepth}{0}
% Redefines (sub)paragraphs to behave more like sections
\ifx\paragraph\undefined\else
\let\oldparagraph\paragraph
\renewcommand{\paragraph}[1]{\oldparagraph{#1}\mbox{}}
\fi
\ifx\subparagraph\undefined\else
\let\oldsubparagraph\subparagraph
\renewcommand{\subparagraph}[1]{\oldsubparagraph{#1}\mbox{}}
\fi

% set default figure placement to htbp
\makeatletter
\def\fps@figure{htbp}
\makeatother


\date{}

\begin{document}

\name{Will Angley}
\address{\\[-9pt] will@willangley.org}
\begin{resume}

\subsubsection{Software Engineer}\label{software-engineer}

\emph{Google Fiber} November 2014 -- February 2017, New York City

\begin{itemize}
\item
  Lead engineering of Fiber-managed WiFi guest networks for apartment
  buildings and small businesses. Adapted autoprovisioning software to
  go from idea to pilot in three months. Handled communication with
  product management, sales, and property managers in the pilot
  deployment.
\item
  Develop autoprovisioning software that allows newly powered up TV
  boxes to learn their own WiFi configuration and start playing TV
  immediately.
\item
  Secure autoprovisioning and guest networks so customers can use them
  safely, with
  \href{https://www.chromium.org/chromium-os/developer-guide/chromium-os-sandboxing}{\texttt{minijail}}
  to contain server processes and \texttt{tc} to let roaming TV boxes
  drop unsolicited traffic cheaply.
\item
  Adapt the
  \href{https://gfiber.googlesource.com/vendor/google/platform/+/master/cmds/isostream.c}{\texttt{isostream}}
  network measurement tool to simulate high-definition wireless TV
  streams. Deployed this simulation to Fiber TV subscribers, found
  sufficient bandwidth for wireless TV and critical WiFi driver bugs.
  After fixes shipped, used simulation to verify them in the field.
\item
  Run an ongoing
  \href{https://blog.codinghorror.com/the-ultimate-dogfooding-story/}{dogfood}
  for the Google Fiber wifi router in 20\% time. Set up 300 tech and
  business Googlers with routers, moderated a mailing list to stay in
  touch with them, and investigated and resolved issues that came up.
  Automated orders so I could do this part time; without automation
  running a dogfood this size is a full-time job.
\end{itemize}

20\% projects, July 2014 -- November 2014

\begin{itemize}
\item
  Extend our in-house logs processing utility Turbogrinder (similar in
  spirit to Stackdriver
  \href{https://cloud.google.com/logging/docs/view/logs_based_metrics}{logs-based
  metrics}, but simple enough for one engineer to build and run) to read
  logs published to our QA server. This let us test it more quickly, and
  without using sensitive data access.
\item
  Build distribution support for Turbogrinder, allowing it to summarize
  time series information including device temperature and ping
  round-trip times from devices in the field.
\end{itemize}

\subsubsection{Technical Solutions
Engineer}\label{technical-solutions-engineer}

\emph{Google Cloud Search} July 2013 -- November 2014, New York City

\begin{itemize}
\item
  Remotely diagnose and repair malfunctioning Google Search Appliances.
\item
  Write customer-deployable support scripts to troubleshoot Appliances
  in embedded applications where no network access is available.
\item
  Develop a customer-deployable configuration profile to quiet fan
  operation on Search Appliances, resolving several dozen escalated
  cases that had been previously thought infeasible. This was one of the
  biggest issues at the time; I got two
  \href{https://www.quora.com/What-are-peer-bonuses-at-Google-How-do-they-work}{peer
  bonuses} and a spot bonus for this work.
\item
  Automate support for customers with common problems by extending our
  team's \href{https://www.youtube.com/watch?v=bFHk2wUaCCs}{support AI}
  to handle Search Appliance cases.
\item
  Work with external partners and vendors to ensure successful
  deployments at large government and commercial customers.
\end{itemize}

\subsubsection{Consultant}\label{consultant}

\emph{Booz Allen Hamilton} June 2008 -- July 2013, McLean, VA

\begin{itemize}
\item
  Developed custom mapping software in Python to ingest and visualize
  months of \href{https://www.uscg.mil/acquisition/nais}{NAIS vessel
  movement data} for a US Government client. Reduced the time it takes
  to go from raw input to finished maps from a week to a day.
\item
  Built a LiDAR data warehouse for a US Government client. Developed a
  Python Web application using Django, C++ data processing utilities,
  Celery to schedule runs of these utilities, and an Oracle Spatial
  backend to store raw and processed data.
\item
  Worked with developers of open source utilities
  (\href{https://github.com/CRREL/points2grid}{\texttt{points2grid}},
  \href{https://www.liblas.org/}{libLAS},
  \href{https://www.pdal.io/}{PDAL},
  \href{http://mapserver.org/}{MapServer},
  \href{http://www.gdal.org/}{GDAL}) used in the data warehouse, and
  contributed patches upstream.
\item
  Rapidly prototyped Asset Management web applications. Generated
  project overview presentations automatically from the database using
  the \href{http://www.reportlab.com/opensource/}{ReportLab Toolkit}.
  Visualized financial data using \href{https://d3js.org/}{D3.js}.
\end{itemize}

\subsubsection{Internships}\label{internships}

\emph{USPS OIG}, June 2007 -- August 2007, Arlington, VA

\emph{Computer Sciences Corporation}, June 2006 -- August 2006,
Chantilly, VA

\emph{USPS IT}, June 2005 -- August 2005, Washington, DC

\subsubsection{Education}\label{education}

M.S., Computer Science, George Washington University, 2013

B.S., Computer Science, College of William and Mary, \emph{magna cum
laude,} 2008

\end{resume}

\end{document}
